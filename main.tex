\documentclass{article}
\usepackage[utf8]{inputenc}
\usepackage[english]{babel}
\usepackage{amsmath}

\usepackage{blindtext}
\usepackage{amssymb}

\usepackage[nottoc]{tocbibind}

\usepackage{amsthm}

\newtheorem{theorem}{Theorem}[section]
\newtheorem{corollary}{Corollary}[theorem]
\newtheorem{lemma}[theorem]{Lemma}
\newtheorem{example}{Example}[section]

\theoremstyle{remark}
\newtheorem*{remark}{Remark}

\theoremstyle{definition}
\newtheorem{definition}{Definition}[section]


\title{An Elementary Approach to Generalised Hone Series}
\author{James Grant Dolan}

\begin{document}

\maketitle

\tableofcontents

\begin{abstract}
    %In a paper published in 2015, A Hone considered a family of integer sequences, with the initial terms $x_0 = x_1 = 1$ generated by non-linear recurrence relations of the form $x_n x_{n-2} = x_{n-1}^2 F(x_{n-1})$ where $F(x)$ is a some particular polynomial. 
    In 2015 \cite{hone2015curious}, Hone considered a family of integer sequences, generated by non-linear recurrence relations of the second order. In this paper we will examine the Hone series generated by taking the sum over the reciprocals of the terms of these sequences. These series have the surprising property that the original sequence interlaced with well defined multiples of the first intersequence form the continued fraction expression of the series. Using lower bounds on the growth rate of the series, we can also determine that the value of the series is transcendental.
\end{abstract}
\newpage

\section*{Introduction}
\addcontentsline{toc}{section}{Introduction}

% Old Intro
\iffalse
In 2015, Andrew N. W. Hone considered a family of integer sequences \cite{hone2015curious} generated by non-linear recurrence relations of the form
\begin{equation}
    x_n x_{n-2} = x_{n-1}^2 F(x_{n-1})
\end{equation}
where $F(x)$ is a polynomial. We will see that this recurrence relation possesses the Laurent property, when thought of as acting on a cluster algebra and as such, by carefully choosing initial variables we can ensure all members of the sequence lie within the integer ring.

In this paper, I will endeavour to present these findings in a transparent and accessible interpretation, using concrete examples, to promote insight and allow for a deeper understanding of the methodology and reasoning. Where appropriate, direct proofs will be sidelined and theorems will instead be justified by making comparisons to material more familiar to the reader. I have also endeavoured to bring together complementary ideas from the original paper and combine them in an elegant way.

From this one root idea, many interesting patterns can be observed. (longer paragraph)

The first of these .. when we add the reciprocals of the sequence and consider the infinite sum $\sum_{i=0}^\infty x_i^{-1}$. We find that through continued fractions expression we can more readily make observations about the nature of the sum and give rigorously proven expressions for the terms that make up the ..

Secondly, using the continued fraction expression we will clearly see that the sum cannot be rational. We will introduce and justify Roth's theorem to show that the infinite sum can not be sufficiently approximated and must therefore be transcendental. To do this we shall use arguments about the growth rate of the terms of the series. This shall be made more rigorous later as we give lower bounds for the growth rate for various cases.

Finally, in the (Questions), we will go on to explore further ideas that could lead on from this initial observation.
\fi

% New Intro
In 2015, Andrew N. W. Hone published the paper, \textit{Curious continued fractions, nonlinear recurrences and transcendental numbers} \cite{hone2015curious}. In the paper Hone identified considerable interest in rational recurrences which generate integer sequences. In particular the recurrence relations popularised by David Gale \cite{gale1991strange} of the form
\begin{equation}
    x_{n+N} x_{n} = f(x_{n+1},\dots,x_{n+N-1})
\end{equation}
where $f$ is a polynomial in $N - 1$ terms. What is of particular interest is that these recurrence relations possess the Laurent property, when thought of as acting on a cluster algebra, and as such by carefully choosing initial variables we can ensure all members of the sequence lie within a given integer ring.

This lead him to consider a much simpler family of integer sequences generated by non-linear recurrence relations of the form
\begin{equation}
    x_{n+2} x_{n} = x_{n+1}^2 F(x_{n+1})
\end{equation}
with initial terms $x_0=x_1=1$. Curiously, although the calculation of these terms involved a final division, where the divisor is not obviously a factor of the dividend, all terms in the sequence are integers. The choice of initial terms guarantees this for any choice of $F$, where $F(0)=1$.
%Here the exchange polynomial is in only one term with a fixed $x^2$ factor.

From this Hone then constructed a specific type of Engel series (later known as a Hone series) consisting of the sum of the reciprocals of the terms of the sequence. By introduction of the continued fraction expression of the series, he was able to more readily make observations about the nature of the series.

In this paper, I will endeavour to present his findings in a transparent and accessible manner, using concrete examples, to promote insight and allow for a deeper understanding of the methodology and reasoning. Where appropriate, direct proofs will be deferred and theorems will instead be justified by making comparisons to material more familiar to the reader. I have also endeavoured to bring together complementary ideas from the original paper and combine them in an elegant way.

Firstly, we will observe that the continued fraction expression is made up of the original sequence interlaced with integer multiples of the first intersequence. We will reason an expression for the odd and even terms of the continued fraction and using proven facts about its convergents prove this expression.

Secondly, using the continued fraction expression we will clearly see that the sum cannot be rational. We will introduce and justify Roth's theorem to show that the infinite sum can not be sufficiently approximated by rational numbers and must therefore be transcendental. To do this we shall use arguments about the growth rate of the terms of the series. This shall be made more rigorous later as we give lower bounds for the growth rate for various cases.

Finally, we will pose some open questions and discuss where Hone and his contemporaries developed these ideas further.

%(not delve into cluster algebras, use conclusions drawn from?)

\newpage

\section{Defining the Sequence}

In this section, we will define what it means for a sequence to belong to this family of sequences, state a few common properties and define its Hone series. We will also introduce helpful notation that enables us to better understand a given sequence's Hone series.

\subsection{Conditions on the Sequence} \label{conditions}
For a sequence $(x_k)_{k\in\mathbb{N}}$ to belong to the family of sequences we are interested in it must satisfy the following conditions;
\begin{enumerate}
\begin{subequations}
    \item The initial terms
    \begin{equation}\label{conditions.initial}
        x_0 = x_1 = 1
    \end{equation}
    \item Further terms are defined by the recurrence relation
    \begin{equation}\label{conditions.recurrence}
        x_n x_{n-2} = f(x_{n-1}) = x_{n-1}^2F(x_{n-1})
    \end{equation}
    \item $F$ is a polynomial of degree\footnote{We restrict degree to exclude the trivial case $F(x) = 1$} $N \geq 1$ of the form
    \begin{equation}\label{conditions.polynomial}
        F(x) = \sum_{i=1}^N b_i x^i + 1
    \end{equation}
    where $b_n \in \mathbb{N}$ for all $n \in \mathbb{N}$
\end{subequations}
\end{enumerate}

The first condition is familiar as it appears in the conditions used to generate the Fibonacci sequence but here plays a much more pivotal role. As can be seen in Hone's paper regarding the Laurent property \cite{hone2007singularity}, the recurrence \eqref{conditions.recurrence} satisfies the Laurent property\footnote{Specifically the third type found in Hone's paper}. This ensures the members of the sequence are Laurent polynomials in the coefficients $x_0^{\pm1}$, $x_1^{\pm1}$ i.e.\ we have $x_n \in \mathcal{R} = \mathbb{Z}[x_0^{\pm1} ,x_1^{\pm1}]$ for all $n \in \mathbb{N}$. By our choice of initial conditions this simplifies to $\mathcal{R} = \mathbb{Z}$ making this a family of integer sequences. If for some $n \in \mathbb{N}$ we have $x_{n} = 0$ then we would arrive at a singularity at the calculation of $x_{n+2}$ and as such, it would be undefined. However, as $F(x) > 0$ the only way we could arrive at a singularity is if we had already encountered a singularity at the previous stage of computation i.e. $x_n = 0 \implies x_{n-1} = 0$. As neither initial term are equal to $0$, we can ensure that this is never the case.
\newpage
\subsection{Intersequences}\label{intersequences}
From any sequence of this form, we can define a further two sequences that share this integral property.
\begin{definition}\label{intersequences.definition}
Take $(x_k)_{k\in\mathbb{N}}$ to be a sequence satisfying the conditions found in section \ref{conditions}. Let $(y_k)_{k\in\mathbb{N}}$ and $(z_k)_{k\in\mathbb{N}}$ be sequences given by
\begin{equation}
    y_n = \frac{x_{n+1}}{x_n} \quad z_n = \frac{y_{n+1}}{y_n}
\end{equation}
We call these the \textbf{first} and \textbf{second intersequences} of $(x_k)$ respectively.
\end{definition}
As with the original sequence, as the calculation of the terms involves a division, it is not immediately clear that these will always form integer sequences as preferred. When trying to prove this it is, in fact, easier to show $(z_k)$ is an integer sequence. If we substitute the one definition into the other we can find an expression for each $z_n$ in terms of the original sequence $(x_k)_{k\in\mathbb{N}}$
\begin{equation}\label{intersequences.substitution}
    z_n = \frac{y_{n+1}}{y_n} = \frac{x_{n+2} x_n}{x_{n+1}^2} = \frac{f(x_{n+1})}{x_{n+1}^2} = F(x_{n+1})
\end{equation}
From this it is clear to see $z_n \in \mathbb{Z}$ for all $n \in \mathbb{N}$.

As we have $y_{n+1} = y_n z_n$, it is clear to see that it can be proved inductively that
\begin{equation}
    y_{n+1} = y_0 \prod_{i=0}^{n} z_i = \prod_{i=0}^{n} z_i \in \mathbb{Z}
\end{equation}
that is $(y_n)$ is an integer sequence also.

In fact, as we also have $x_{n+1} = x_n y_n$ we can prove
\begin{equation}\label{intersequences.yproduct}
    x_{n+1} = \prod_{i=0}^n y_i \in \mathbb{Z}
\end{equation}
and thus prove directly that $(x_k)$ is an integer sequence.

As the initial terms of the sequence $(x_k)_{k\in\mathbb{N}}$ are known, we can give more specific definitions for the first few terms of these intersequences.
\begin{equation}\label{intersequences.evaluation}
\begin{split}
    y_0 &= \frac{x_1}{x_0} = \frac{1}{1} = 1, \quad y_1 = \frac{x_2}{x_1} = \frac{x_2}{1} = x_2 \\
    z_0 &= \frac{y_1}{y_0} = \frac{x_2}{1} = x_2
\end{split}
\end{equation}
Using the previously derived formula, we can combine \eqref{intersequences.substitution} and \eqref{intersequences.evaluation} to give
\begin{equation}\label{intersequences.combination}
    x_2 = y_1 = z_0 = F(x_1) = F(1)
\end{equation}

\subsection{Examples}\label{examples}
Using an example directly from Hone's paper, I shall demonstrate my claims and go through the workings as follows.
\begin{example}\label{examples.1}
Let $(x_k)_{k\in\mathbb{N}}$ be a sequence under the conditions found in section \ref{conditions} where
\begin{equation}
    F(x) = x + 1
\end{equation}
First, consider the first few terms generated by this sequence.
\begin{align*}
    (x_k)&: 1,1,2,12,936,68408496,342022190843338960032,\dots
\intertext{As expected these terms are all integers. If we then go on to generate terms of $(y_n)$ using the terms generated so far, we see these are also integers}
(y_k)&: 1,2,6,78,73086,4999703411742,\dots
\intertext{As are the terms of $(z_n)$}
(z_k)&: 2,3,13,937,68408497\dots
\end{align*}
Evaluating $F(x_5)$ we can show 
\begin{equation*}
    F(x_5) = F(68408496) = 68408496 + 1 = 68408497 = z_4
\end{equation*}
This satisfies the identity \eqref{intersequences.substitution}.
It is clear from its definition that $F(1) = 3$. This satisfies the identity \eqref{intersequences.combination} as we have $x_2 = y_1 = z_0 = 3$.
\end{example}
Now, using a similar example of my own construction, I hope to further demonstrate and reiterate the mechanics of the claims made.
\begin{example}\label{examples.2}
Let $(x_k)_{k\in\mathbb{N}}$ be a sequence under the conditions found in section \ref{conditions} where
\begin{equation}
    F(x) = x^2 + x + 1
\end{equation}
Again, we calculate the first few terms of the sequence and its two intersequences.
\begin{align*}
    (x_k)&: 1,1,3,117,63001341,134652158504237974615172605539,\dots \\
    (y_k)&: 1,3,39,538473,2137290355521765395679,\dots \\
    (z_k)&: 3,13,13807,3969169030799623,\dots
\end{align*}
As seen previously, these sequences involved only integers so far, as would be expected.

Evaluating $F(x_4)$ we can show 
\begin{align*}
    F(x_4) &= F(63001341) \\
    &= 63001341^2 + 63001341 + 1 \\
    &= 3969168967798281 + 63001341 + 1 \\
    &= 3969169030799623 = z_3
\end{align*}
Once more, this satisfies the identity \eqref{intersequences.substitution}.
It is clear to see that $F(1) = 3$. This satisfies the identity \eqref{intersequences.combination}.
\end{example}

We shall return to these examples later to demonstrate further properties in practice.

\subsection{The Hone Series}\label{sum}
Perhaps the most interesting quantity to derive from these sequences is the sums of the reciprocals of the terms. That is 
\begin{equation}
    \mathcal{S} = \sum^\infty_{i=0}\frac{1}{x_i}
\end{equation}
It feels natural to take this sum over the entire sequence. However, if we omit the constant first term, the sum has much more desirable properties as we will show later.
\begin{definition}
Let $(x_k)_{k\in\mathbb{N}}$ be a sequence under the conditions found in section \ref{conditions}. We define
\begin{equation}
    S_N = \sum^N_{i=1}\frac{1}{x_i}
\end{equation}
We call this the sequence's \textbf{partial Hone series}.

We can define $S_\infty$ by taking the limit $N \to \infty$.
\begin{equation}\label{sum.infty}
    S_\infty = \lim_{N \to \infty} S_N = \sum^\infty_{i=1}\frac{1}{x_i}
\end{equation}
We call this the sequence's \textbf{infinite} Hone series, though often we will omit the infinite.
\end{definition}
We can justify the definition of $S_\infty$ as the sequence grows very rapidly. In fact we know $x_n > n^2$ for all $n \geq 3$ as this is true in the case of example \ref{examples.1} and it is clear to see that this is the slowest growing sequence we can generate given our constraints.

Another way to express this series is by rewriting each addend of the series as a product of terms in $(y_n)$. Doing so we obtain the following;
\begin{align*}
    S_\infty &= \sum_{i=0}^\infty \prod_{j=1}^{i} \frac{1}{y_j}
    \intertext{We can clean this up by separating the case $i=0$, where there are no multipliers in the product, so it can be written without pi notation.}
    S_\infty - 1 &= \sum_{i=1}^\infty \frac{1}{y_1y_2\dots y_i}
\end{align*}
As we have $y_n \in \mathbb{N}_{\geq 2}$ for all $n \in \mathbb{N}$ and $(y_n)$ non decreasing the series, we can see that the Hone series are a type of Engel series \cite{duverney2010number}.

\begin{example}\label{sum.example.1}
Take the sequence, $(x_k)_{k\in\mathbb{N}}$, given in example \ref{examples.1}. Now we compute the first few partial Hone series.
\begin{alignat*}{3}
    & S_3 &&= \frac{1}{1} + \frac{1}{2} + \frac{1}{12} &&= \frac{19}{12} \\
    & S_4 &&= \frac{1}{1} + \frac{1}{2} + \frac{1}{12} + \frac{1}{936} &&= \frac{1483}{936} \\
    & S_5 &&= \frac{1}{1} + \frac{1}{2} + \frac{1}{12} + \frac{1}{936} + \frac{1}{68408496} &&= \frac{108386539}{68408496}
\end{alignat*}
Interestingly, we see that so far the denominator of the final summand is equal to that of the entire series. This is because, by the integrity of $(y_n)$, any term in the sequence is divisible by any term that precedes it.
\end{example}

\begin{example}\label{sum.example.2}
Take the sequence, $(x_k)_{k\in\mathbb{N}}$, given in example \ref{examples.2}. Again, we compute the first few partial Hone series.
\begin{alignat*}{3}
    & S_2 &&= \frac{1}{1} + \frac{1}{3} &&= \frac{4}{3} \\
    & S_3 &&= \frac{1}{1} + \frac{1}{3} + \frac{1}{117} &&= \frac{157}{117} \\
    & S_4 &&= \frac{1}{1} + \frac{1}{3} + \frac{1}{117} + \frac{1}{63001341} &&= \frac{84540262}{63001341}
\end{alignat*}
As seen previously, the sum inherits the denominator of its final term.
\end{example}

\subsection{Continued Fractions}\label{cfrac}

As we know, a general finite continued fraction of length $n$ can be written as follows;
\begin{equation}
    [a_0;a_1,a_2,\dots,a_n] = a_0 + \cfrac{1}{a_1 + \cfrac{1}{a_2 + \cfrac{1}{\ddots + \cfrac{1}{a_n}}}}
\end{equation}
If the continued fraction is infinite, we write
\begin{equation}
    [a_0;a_1,a_2,\dots] = a_0 + \cfrac{1}{a_1 + \cfrac{1}{a_2 + \ddots}}
\end{equation}

Given an infinite continued fraction we can also take its convergents. We write a continued fractions $m$th convergent as follows;
\begin{equation}
    \frac{p_m}{q_m} = [a_0;a_1,a_2,\dots,a_m]
\end{equation}
where each $a_i$ is taken from the infinite continued fraction. So that this expression is unique we assume that this is a simplified rational expression with positive denominator. The value of the continued fraction can therefore be defined as the limit of its convergents. We can also take convergents of a finite continued fraction provided $m \leq n$. Note that the $n$th convergent is simply the continued fraction in its simplified form.

\begin{example}\label{cfrac.example.1}
Take the sequence, $(x_k)_{k\in\mathbb{N}}$, given in example \ref{examples.1}. Now we write the previously calculated partial Hone series from example \ref{sum.example.1} in continued fraction notation. First lets manually calculate $S_3$.
\begin{align*}
    S_3 = \frac{19}{12} &= 1 + \frac{7}{12} = 1 + \cfrac{1}{12/7} = 1 + \cfrac{1}{1+\cfrac{5}{7}} \\
    &= 1 + \cfrac{1}{1+\cfrac{1}{7/5}} = 1 + \cfrac{1}{1+\cfrac{1}{1+\cfrac{2}{5}}} \\
    &= 1 + \cfrac{1}{1+\cfrac{1}{1+\cfrac{1}{5/2}}} = 1 + \cfrac{1}{1+\cfrac{1}{1+\cfrac{1}{2+\cfrac{1}{2}}}} \\
    &= [1;1,1,2,2]
\end{align*}
Now lets compute the next two sums.
\begin{align*}
    S_4 &= [1;1,1,2,2,6,12] \quad S_5 = [1;1,1,2,2,6,12,78,937]
\end{align*}
Notice how when looking at the continued fraction expression for successive sums, the former is the prefix for the latter. In fact, each partial series is an expansion of the one before.
The even terms\footnote{when indexed from zero} of each of the continued fractions form the terms of the sequence, $(x_k)$. Likewise, the odd terms form the terms of the first intersequence, $(y_k)$.
\end{example}

\begin{example}\label{cfrac.example.2}
Take the sequence, $(x_k)_{k\in\mathbb{N}}$, given in example \ref{examples.2}. Again we write out the partial series from \ref{sum.example.2} in continued fraction notation.
\begin{align*}
    S_3 &= [1;2] = [1;1,1] \quad S_4 = [1;2,1,12,3] \\
    S_5 &= [1;2,1,12,3,4602,117]
\end{align*}
Again, we see the original sequence in the even terms of the expression. This time however the odd terms are not exactly the first intersequence though they do appear as multiples of their corresponding in the intersequence. This can be demonstrated by rewriting $S_5$.
\begin{equation}
    S_5 = [1;2(1),1,4(3),3,118(39),117] = [x_0;2y_0,x_1,4y_1,x_2,118y_2,x_3]
\end{equation}
While it is unclear at this time where these coefficients come from, in this case it is interesting to note that they seem to be of the form $(x_{n+1} + 1)$ where $x_{n+1}$ is the next term in the sequence.

It is important to note that, for this property to be seen in $S_3$, we had to write the continued fraction in its non-canonical form. However, if the pattern observed thus far is true for all Hone series, this will be a unique case only seen in partial Hone series of this index.
\end{example}

\section{Terms of the Continued Fraction}
In this section, we will endeavour to use the already understood quantities to find representation for the Hone series when in continued fraction expression. To do this, we will find generalised representations for the terms of the continued fraction.

\subsection{Hone's Correspondence Theorem}\label{correspondence}
We can see from the examples in \ref{cfrac} that there is a clear correspondence between the terms of a sequence and the terms of the continued fraction expression for the sequence's Hone series. Namely, in every case thus far encountered the sub sequence formed by the even terms of a Hone series' continued fraction expression is exactly the sequence used to generate it.

Less clear however is how the odd terms factor into things. In example \ref{cfrac.example.1}, where $F(x) = x + 1$, the subsequence formed by the odd terms of the continued fraction expression was exactly the first intersequence. From this we can infer a dependence on $x_n$, $x_{n+1}$, and $F$. If $F$ appears in the expression, we would also expect to find it in the form of either $F(x-1)$ or $F(x)-1$ so as it cancels in the example case, where there is no implicit dependence on F in the odd terms.

In example \ref{cfrac.example.2}, where $F(x) = x^2 + x + 1$. We find $F(x-1) = x^2 - x + 1$ which contains an undesirable constant term. If we instead consider $F(x) - 1$, we find
\begin{align*}
    F(x) - 1 &= x^2 + x = x(x + 1)
    \intertext{which under the substitution $x = x_{n+1}$ becomes}
    F(x_{n+1}) - 1 &= x_{n+1}^2 + x_{n+1} = x_{n+1}(x_{n+1} + 1).
    \intertext{The factor $x_{n+1} + 1$ is exactly the coefficient of $y_n$ hypothesised in the example. Taking this further, we can divide both sides by $x_n$ to obtain}
    \frac{F(x_{n+1}) - 1}{x_n} &= \frac{x_{n+1}(x_{n+1} + 1)}{x_n} = y_n(x_{n+1} + 1)
\end{align*}
which agrees exactly with what was seen in the example.

By Hone's paper we know this to be the correct formula, so for our purposes we can take this as our final formula and move onto the theorem.

\begin{theorem}\label{correspondence.theorem}
Let $(x_k)_{k\in\mathbb{N}}$ be a sequence under the conditions found in section \ref{conditions}. The partial Hone series when expressed as a continued fraction are of the form
\begin{equation}
    S_N = [a_0,a_1,a_2,\dots,a_{2N -2}]
\end{equation}
for all $N \geq 1$, where
\begin{equation}\label{correspondence.terms}
    a_{2n} = x_n, \quad a_{2n+1} = \frac{F(x_{n+1})-1}{x_n}
\end{equation}
\end{theorem}
Leading on from this, we can take the limit of the partial Hone series and make a statement about the infinite Hone series, using already understood properties of limits of continued fractions.
\begin{corollary}\label{correspondence.corollary}
Let $(x_k)_{k\in\mathbb{N}}$ be a sequence under the conditions found in section \ref{conditions}. The Hone series when expressed as a continued fraction is of the form
\begin{equation}
    S_\infty = [a_0,a_1,a_2,\dots]
\end{equation}
where each $a_k$ is defined as in \eqref{correspondence.terms}.
\end{corollary}

\iffalse
Note as a result of this theorem, we can write $z_n$ as a function of $a_{2n-1}$.
\begin{equation}
    z_n = F(x_n) = a_{2n-1}x_{n-1} + 1 = G(a_{2n-1})
\end{equation}

In practice, we can see that this means the odd terms are always integer multiples of the first intersequence as using the form of $F$ given in \eqref{conditions.polynomial} they can be rewritten
\begin{align*}
    \frac{F(x_{n+1})-1}{x_n} &= \frac{1}{x_n} \sum_{i=1}^N b_i x_{n+1}^i = \frac{x_{n+1}}{x_n} \sum_{i=1}^N b_i x_{n+1}^{i - 1} \\
    &= y_n \sum_{i=1}^N b_i x_{n+1}^{i - 1} = y_n \sum_{i=0}^N b_{i+1} x_{n+1}^i
\end{align*}
\fi
Notice that by using \eqref{conditions.polynomial} we can write $F(x) = x G(x) + 1$ for some $G(x) \in \mathbb{Z}[x]$. Combining this with \eqref{correspondence.terms} we arrive at
\begin{align*}
    a_{2n+1} = \frac{F(x_{n+1})-1}{x_n} &= \frac{x_{n+1} G(x_{n+1})}{x_n} = y_n G(a_{2n+2})
\end{align*}
Therefore the coefficient will always be defined by some polynomial in the successive term of the continued fraction.

\begin{example}
Take the sequence, $(x_k)_{k\in\mathbb{N}}$, given in example \ref{examples.1}. Using theorem \ref{correspondence.theorem}, we find the terms of this Hone series are;
\begin{equation*}
    a_{2n} = x_n, \quad a_{2n+1} = \frac{(x_{n+1} + 1) - 1}{x_n} = \frac{x_{n+1}}{x_n} = y_n
\end{equation*}
This then produces the continued fraction $S_\infty = [x_0,y_0,x_1,y_1,\dots]$ which is exactly what was observed in the previous example.
\end{example}

\begin{example}
Take the sequence, $(x_k)_{k\in\mathbb{N}}$, given in example \ref{examples.2}. Using theorem \ref{correspondence.theorem}, we find the terms of this Hone series are;
\begin{equation*}
    a_{2n} = x_n, \quad a_{2n+1} = \frac{(x_{n+1}^2 + x_{n+1}+1) - 1}{x_n} = \frac{x_{n+1}^2 + x_{n+1}}{x_n}
\end{equation*}
It is obvious that this pattern in the even terms was seen in the example, however we will need to calculate the corresponding odd terms to verify agreement with the directly calculated values.
\begin{align*}
    a_1 &= \frac{x_{1}^2 + x_{1}}{x_0} = \frac{1^2 + 1}{1} = \frac{2}{1} = 2 = 2 \cdot 1 = 2y_0 \\
    a_3 &= \frac{x_{2}^2 + x_{2}}{x_1} = \frac{3^2 + 3}{1} = \frac{12}{1} = 12 = 4 \cdot 3 = 4y_1 \\
    a_5 &= \frac{x_{3}^2 + x_{3}}{x_2} = \frac{117^2 + 117}{3} = \frac{13806}{3} = 4602 = 118 \cdot 39 = 118y_2
\end{align*}
After interlacing these between the even terms we obtain
\begin{align*}
    S_5 = [x_0;2y_0,x_1,4y_1,x_2,118y_2,x_3]
\end{align*}
as was seen in example \ref{cfrac.example.2}.
\end{example}

\subsection{Convergents of the Series}

% Could go in the continued fraction subsection?
To be able to perform algebra on the individual rational components of the convergents rather than the ratio, we will need to understand the recurrence relations governing them.
\begin{equation}\label{convergents.indentity.linear}
    \begin{split}
        p_{n+1} = a_{n+1}p_n + p_{n-1} \\
        q_{n+1} = a_{n+1}q_n + q_{n-1}
    \end{split}
\end{equation}
It is clear that the initial values are
\begin{alignat*}{2}
    p_0 &= a_0,\quad &&q_0 = 1
    \intertext{To make this recursive definition well defined we will need to also provide values for an adjacent index. We can easily calculate the values at index $1$.}
    p_1 &= a_0 a_1 + 1,\quad &&q_1 = a_1
    \intertext{While these initial values will satisfy these recurrence relations, it is cleaner for us to give the values at index $-1$.}
    p_{-1} &= 1,\quad &&q_{-1} = 0
\end{alignat*}
Though this does not mean we can have the $-1$th convergent of a continued fraction, allowing this definition does simplify much of the calculations going forward.

We can re-frame this relation as a matrix identity with the inclusion of the trivial identity of $p_n$ and $q_n$.
\begin{equation}\label{convergents.indentity.matrix}
    \begin{pmatrix}
    p_{n+1} & q_{n+1}\\
    p_{n} & q_{n}
    \end{pmatrix}
    =
    \begin{pmatrix}
    a_{n+1} & 1\\
    1 & 0
    \end{pmatrix}
    \begin{pmatrix}
    p_{n} & q_{n}\\
    p_{n-1} & q_{n-1}
    \end{pmatrix}
\end{equation}

% Identity lemma
\begin{lemma}\label{convergents.lemma.1}
Let $(p_k/q_k)_{k \in \mathbb{N}}$ be the convergents of some continued fraction $[a_0;a_1,a_2,\dots]$. For all $n \in \mathbb{N}$ we have
\begin{subequations}
\begin{equation}\label{convergents.identity.1}
    p_nq_{n-1} - p_{n-1}q_n = (-1)^{n+1}
\end{equation}
Following on from this we also have
\begin{equation}\label{convergents.identity.2}
    p_nq_{n-2} - p_{n-2}q_n = (-1)^n a_n
\end{equation}
\end{subequations}
\end{lemma}
\begin{proof}
Through repeated use of the matrix identity \eqref{convergents.indentity.matrix} we can find an expression for the matrix in terms of matrices made from constants and terms of the continued fraction.
\begin{align*}
    \begin{pmatrix}
    p_{n} & q_{n}\\
    p_{n-1} & q_{n-1}
    \end{pmatrix}
    =
    \begin{pmatrix}
    a_{n} & 1\\
    1 & 0
    \end{pmatrix}
    \begin{pmatrix}
    a_{n-1} & 1\\
    1 & 0
    \end{pmatrix}
    \dots
    \begin{pmatrix}
    a_{0} & 1\\
    1 & 0
    \end{pmatrix}
\end{align*}
If we take the determinant of both sides we arrive at \eqref{convergents.identity.1} as required.

Following on from this, we find that
\begin{align*}
    (-1)^n a_n &= -a_n(p_n q_{n-1} - p_{n-1}q_n) \\
    &= a_n(p_{n-1}q_n - p_n q_{n-1}) \\
    &= a_n p_{n-1} q_n - a_n p_n q_{n-1} 
    \intertext{Using \eqref{convergents.indentity.linear} we can eliminate the explicit dependence on $a_n$ and the $n-1$ indexed rational components.}
    (-1)^n a_n &= (p_{n} - p_{n-2}) q_n - p_n (q_{n} - q_{n-2}) \\
    &= p_n q_{n-2} - p_{n-2}q_n
\end{align*}
as required.
\end{proof}

% Convergent denominator lemma
\begin{lemma}\label{convergents.lemma.2}
Let $(p_k/q_k)_{k \in \mathbb{N}}$ be the convergents of some continued fraction. If for some $n \in \mathbb{N}$ we have 
\begin{equation}
    \frac{p_{2n}}{q_{2n}} = [a_0;a_1,\dots,a_{2n}]
\end{equation}
where the terms $a_k$ satisfy \eqref{correspondence.terms}, then we have
\begin{equation}\label{convergents.denominator}
    q_{2n-1} = y_{n} - 1,\quad q_{2n} = x_{n+1}
\end{equation}
\end{lemma}
\begin{proof}
In the case that $n = 0$, \eqref{convergents.denominator} is always satisfied, as
\begin{align*}
    y_{0} - 1 = 0 = q_{-1},\quad x_1 = 1 = q_0.
\end{align*}
and as such the lemma technically holds.

Suppose the lemma for some $n=N$ i.e. if $p_{2N}/q_{2N}$ has terms \eqref{correspondence.terms} then we have \eqref{convergents.denominator} for $n=N$.

Assume $p_{2N+2}/q_{2N+2}$ has terms \eqref{correspondence.terms}. The conditions for the supposed lemma are then satisfied as the lower order convergent, $p_{2N}/q_{2N}$, will share the same terms. Starting with the convergent formula for $q_{2N+1}$ from \eqref{convergents.indentity.linear}

\begin{align*}
    q_{2N+1} &= a_{2N+1}q_{2N} + q_{2N-1} 
    \intertext{We can use the substitutions found in  \eqref{correspondence.terms} and \eqref{convergents.denominator} to obtain}
    q_{2N+1} &= a_{2N+1}x_{N+1} + y_{N} - 1 \\
    &= \frac{F(x_{N+1})-1}{x_{N}}x_{N+1} + y_{N} - 1 \intertext{We then rewrite this in terms of clear factors of $(x_n)$.}
    q_{2N+1} &= (F(x_{N+1})-1)\frac{x_{N+1}}{x_{N}} + \frac{x_{N+1}}{x_{N}} - 1 \\
    &= F(x_{N+1})\frac{x_{N+1}}{x_{N}} - 1 
    \intertext{We then put this back in terms of $(y_n)$ and $(z_n)$ to obtain}
    q_{2N+1} &= z_{N} y_{N} - 1 = y_{N+1} - 1
    \intertext{This completes the first half of our inductive step. We can then go on to express $q_{2N+2}$ in a similar way. We again start with}
    q_{2N+2} &= a_{2N+2}q_{2N+1} + q_{2N} 
    \intertext{We then use our previous result together with \eqref{correspondence.terms} and \eqref{convergents.denominator} to find}
    q_{2N+2} &= x_{N+1}(y_{N+1}-1) + x_{N+1} \\
    &= x_{N+1}y_{N+1} - x_{N+1} + x_{N+1} \\
    &= x_{N+2}
\end{align*}
This implies the theorem holds for $n = N + 1$ and thus, by induction we can infer that the theorem holds for all $n \geq 0$.
\end{proof}

\iffalse
\begin{lemma}
Let $(p_k/q_k)_{k \in \mathbb{N}}$ be the convergents of some continued fraction series, $S_M$. For all $n \leq M$, given $(a_k)_{k \leq 2n}$ satisfy \eqref{correspondence.terms}, where $a_k$ is the $k$th term of the continued fraction, we have
\begin{equation}\label{convergents.denominator}
    q_{2n-1} = y_{n} - 1,\quad q_{2n} = x_{n+1}
\end{equation}
\end{lemma}
\begin{proof}
In the case that $n = 0$, without reference to $a_0$ we have
\begin{align*}
    y_{0} - 1 = 0 = q_{-1},\quad x_1 = 1 = q_0
\end{align*}
which agrees with the lemma.

If we suppose the theorem holds for some $n = N$, then, assuming $(a_k)_{k \leq 2N}$, $a_{2N+1}$, $a_{2N+2}$ satisfy \eqref{correspondence.terms}, we can express $q_{2N+1}$ as follows;
\begin{align*}
    q_{2N+1} &= a_{2N+1}q_{2N} + q_{2N-1} \\
    &= a_{2N+1}x_{N+1} + y_{N} - 1 
    \intertext{Then using our auxiliary assumption regarding \eqref{correspondence.terms} we obtain}
    q_{2N+1} &= \frac{F(x_{N+1})-1}{x_{N}}x_{N+1} + y_{N} - 1 \intertext{We then rewrite this in terms of clear factors of $(x_n)$.}
    q_{2N+1} &= (F(x_{N+1})-1)\frac{x_{N+1}}{x_{N}} + \frac{x_{N+1}}{x_{N}} - 1 \\
    &= F(x_{N+1})\frac{x_{N+1}}{x_{N}} - 1 
    \intertext{We then put this back in terms of $(y_n)$ and $(z_n)$ to obtain}
    q_{2N+1} &= z_{N} y_{N} - 1 = y_{N+1} - 1
    \intertext{This completes the first half of our inductive step. We can then go on to express $q_{2N+2}$ in the same way.}
    q_{2N+2} &= a_{2N+2}q_{2N+1} + q_{2N} 
    \intertext{We then use our previous result together with the inductive assumption to find}
    q_{2N+2} &= x_{N+1}(y_{N+1}-1) + x_{N+1} \\
    &= x_{N+1}y_{N+1} - x_{N+1} + x_{N+1} \\
    &= x_{N+2}
\end{align*}
This implies the theorem holds for $n = N + 1$ and thus by induction we can infer that the theorem holds for all $n \geq 0$.
\end{proof}
\fi

With those two lemmas, we now have all of the required knowledge to go back and prove Theorem \ref{correspondence.theorem}.
\newpage
\subsection{Proof of the Theorem}
We aim to prove theorem \ref{correspondence.theorem} using induction. We can rewrite each partial series in convergent notation so as we can use lemma \ref{convergents.lemma.2} to perform the inductive step.
\begin{proof}[Proof of Theorem \ref{correspondence.theorem}]
In the case that $N=1$, we have
\begin{align*}
    S_N = S_1 = \frac{1}{x_1} = 1 = [x_0]
\end{align*}
which agrees with the theorem.

If we suppose the theorem holds for some $N = M$, we can express $S_{M+1}$ as follows;
\begin{align*}
    S_{M+1} &= S_M + \frac{1}{x_{M+1}} = [a_0,a_1,a_2,\dots,a_{2M -2}] + \frac{1}{x_{M+1}}
    \intertext{We can think of $S_M$ as a convergent of some infinite continued fraction, $S$, with terms taken from \eqref{correspondence.terms} and rewrite our expression for $S_{M+1}$ in convergent notation using lemma \ref{convergents.lemma.2}.}
    S_{M+1} &= \frac{p_{2M-2}}{q_{2M-2}} + \frac{1}{q_{2M}} = \frac{p_{2M-2}q_{2M} + q_{2M-2}}{q_{2M-2}q_{2M}}
    \intertext{We can use the identity \eqref{convergents.identity.2} to perform a change of index for $p$.}
    S_{M+1} &= \frac{p_{2M-2}q_{2M} + q_{2M-2}}{q_{2M-2}q_{2M}} = \frac{p_{2M}q_{2M-2} - a_{2M} + q_{2M-2}}{q_{2M-2}q_{2M}}
    \intertext{Again, we can use lemma \ref{convergents.lemma.2} together with our initial assumption to obtain}
    S_{M+1} &= \frac{p_{2M}q_{2M-2} - a_{2M} + q_{2M-2}}{q_{2M-2}q_{2M}} = \frac{p_{2M}q_{2M-2} - x_{M} + x_M}{q_{2M-2}q_{2M}}
    \intertext{Through simple cancellation of terms we find this results in}
    S_{M+1} &= \frac{p_{2M}q_{2M-2}}{q_{2M-2}q_{2M}} = \frac{p_{2M}}{q_{2M}}
\end{align*}
We have now found that $S_{M+1}$ is exactly the $2M$th convergent of $S$ and as such also has terms in \eqref{correspondence.terms}. This then implies the theorem holds for $N = M + 1$ and thus by induction we can infer that the theorem holds for all $N \geq 1$.
\end{proof}
\newpage
\section{Transcendence of the Sums}
In this section, we will endeavour to prove the transcendence of the Hone series. As transcendental numbers cannot be too closely approximated by rationals, we will aim to find an appropriate method of testing approximations so as we can determine this.

\subsection{Diophantine Approximation}

We can use Diophantine approximation to test for transcendence in these sums. As each sum can be represented by an infinite non repeating continued fraction we know that they are irrational and thus might want to use something like Liouville's theorem to try to prove this.
\begin{theorem}{Liouville's Theorem}\label{trancsendence.liouville}

    Let $\xi$ be an irrational algebraic number of degree $n$. There exists $c>0$ such that for all $p \in \mathbb{Z}$, $q \in \mathbb{N}$. We have
    \begin{equation}\label{trancsendence.liouville.eq}
        \left| \xi - \frac{p}{q} \right| \geq \frac{1}{cq^n}
    \end{equation}
\end{theorem}

\begin{proof}
    Let $f(x) \in \mathbb{Q}[x]$ be a polynomial of degree $n$ with roots 
    \begin{equation}
        \xi_1=\xi,\xi_2,\dots,\xi_n.
    \end{equation}
    As $\xi$ is not a root of any polynomial of lesser degree all $\xi_i$ are irrational. If this were not the case, a monomial with such a rational root could be factored out of $f$ and $\xi$ would then be a root of the other polynomial factor of degree $n-1$.
    
    Let $F(x) \in \mathbb{Z}[x]$ be a corresponding polynomial such that $F(x) = Q f(x)$ for some $Q \in \mathbb{N}$. For any rational number $p/q \in \mathbb{Q}$ we have
    \begin{align*}
        F\left(\frac{p}{q} \right) = a_n\frac{p^n}{q^n} + \dots + a_1\frac{p}{q} + a_0 = \frac{P}{q^n}
    \end{align*}
    for some $a_n,\dots,a_1,a_0,P \in \mathbb{Z}$. Then, multiplying through by $q^N$ we obtain $q^n F(p/q) = P \in \mathbb{Z}$. We know $P \neq 0$ as otherwise would imply either $F(p/q) = 0$ or $q = 0$. This implies $|q^n F(p/q)| = |P| \geq 1$ and subsequently $|F(p/q)| \geq q^{-n}$ if we take $q \in \mathbb{N}$.
    
    If $|\xi - p/q| \geq 1$ then taking $c < 1$ trivialises the proof and thus without loss of generality we can assume $|\xi - p/q| < 1$. Note that by the triangle inequality we have
    \begin{equation}\label{trancsendence.liouville.triangle}
        |p/q| = |(p/q - \xi) + \xi| < 1 + |\xi|.
    \end{equation}
    By rewriting $F$ in terms of its roots we have
    \begin{align*}
        q^{-n} &\leq |F(p/q)| = |a_n(p/q - \xi)(p/q - \xi_2)\dots(p/q - \xi_n)| \\
        &\leq |a_n||p/q - \xi_2|\dots|p/q - \xi_n||p/q - \xi| \\
        &\leq |a_n|(|p/q| + |\xi_2|)\dots(|p/q| + |\xi_n|)|p/q - \xi|
        \intertext{By \eqref{trancsendence.liouville.triangle} we now have}
        q^{-n} &\leq \big[|a_n|(1 + |\xi| + |\xi_2|)\dots(1 + |\xi| + |\xi_n|)\big]|p/q - \xi|
        \intertext{We then relabel the constant term in our inequality by $c = [\dots]$ to obtain}
        q^{-n} &\leq c|p/q - \xi|
    \end{align*}
    Now all that remains is to bring the constant $c$ over to the other side of the inequality so as we arrive at \eqref{trancsendence.liouville.eq} as required.
\end{proof}
The above is a common statement of Liouville's theorem. We can restate the theorem to allow easier comparison to other approximation theorems.
\begin{corollary}\label{trancsendence.liouville.corollary}
    Let $\xi$ be an irrational algebraic number of degree $n$. There exists $c>0$ such that there are no solutions $p \in \mathbb{Z}$, $q \in \mathbb{N}$ for which
    \begin{equation}\label{trancsendence.liouville.2}
        \left| \xi - \frac{p}{q} \right| < \frac{1}{c q^n}
    \end{equation}
\end{corollary}
%This is the statement of the theorem given in J W S Cassels' book \cite{cassels1957introduction}.
This can be seen as the inverse of theorem \ref{trancsendence.liouville}. Both theorems indicate that an irrational algebraic number cannot be ``too closely" approximated by rational numbers. The idea will prove useful, however Liouville's theorem alone is not strong enough for our purposes. We will need to introduce another theorem to prove transcendence in all cases.
\begin{theorem}{Roth's Theorem}\label{trancsendence.roth}

    Let $\xi$ be an irrational algebraic number. For all $\delta>0$, there exists only finitely many solutions $p \in \mathbb{Z}$, $q \in \mathbb{N}$ for which
    \begin{equation}
        \left| \xi - \frac{p}{q} \right| < \frac{1}{q^{2+\delta}}
    \end{equation}
\end{theorem}

By comparison to \eqref{trancsendence.liouville.2}, this can be seen to be an improvement on Liouville's theorem, most notably due to a lack of a dependence on $n$ in the inequality. Without a great leap from Liouville, the validity of the theorem can be conceived.

A more rigorous proof can be found in Cassels' book on the subject \cite{cassels1957introduction}.

\subsection{A Criterion for Transcendence}

\begin{lemma}\label{trancsendence.criterion}
Let $(x_k)_{k\in\mathbb{N}}$ be a sequence such that for all $n \in \mathbb{N}$, $x_{n+1} > x_n^\kappa \neq 1$ for some $\kappa > 2$. Then $S_\infty$, as found in \eqref{sum.infty}, is irrational exactly when it is transcendental.
\end{lemma}

\begin{proof}
From the growth condition, we can determine $x_{n+i} > x_n^{\kappa^i}$. Using this we have
\begin{align*}
    | S_\infty - S_n | &= \sum_{i=n+1}^\infty \frac{1}{x_{i}} = \sum_{i=1}^\infty \frac{1}{x_{n+i}} < \sum_{i=1}^\infty \frac{1}{x_n^{\kappa^i}} 
    \intertext{Using the fact that $\kappa^i > i\kappa$ for all $j \geq 2$ we then have}
    | S_\infty - S_n | &< \sum_{i=1}^\infty \frac{1}{x_n^{i\kappa}} = \frac{1}{x_n^\kappa} \frac{1}{1-x_n^\kappa} < \frac{x_n^\epsilon}{x_n^\kappa} = \frac{1}{x_n^{\kappa-\epsilon}}
\end{align*}
for all $\epsilon > 0$. We can choose an $\epsilon$ such that $\kappa-\epsilon = 2 + \delta > 2$. which proves each $S_n$ is a ``close enough" approximation for $S_\infty$. Given there are infinitely many $S_n$, if $S_\infty$ is irrational then it must also be transcendental as not to contradict Roth's theorem \eqref{trancsendence.roth}.

As it is trivial to prove that any transcendental number is irrational we know that $S_\infty$ is irrational exactly when it is transcendental.
\end{proof}

While we have found \textit{a} proof of transcendence, it remains to be proved that this lemma applies to the Hone series. That is, we have yet to prove that our sequences satisfy the growth condition, $x_{n+1} > x_n^\kappa$ for some $\kappa > 2$.

\subsection{The Growth Condition}

If we take $(x_k)_{k\in\mathbb{N}}$ to be a sequence satisfying the conditions found in section \ref{conditions}, we have $F(x) \geq b_N x^N \geq x^N$ for all $x \in \mathbb{N}$. Combining this with the recurrence relation gives
\begin{align*}
    x_n &= \frac{x_{n-1}^2 F(x_{n-1})}{x_{n-2}} \geq \frac{(x_{n-1})^{N+2}}{x_{n-2}}
    \intertext{Using the fact that $y_n \geq 1$ for all $n \in \mathbb{N}$ we can obtain}
    x_n &\geq y_{n-2}(x_{n-1})^{N+1} \geq (x_{n-1})^{N+1}
\end{align*}
so it is clear that the criterion found in Lemma \ref{trancsendence.criterion} is satisfied for $N > 1$, $n \geq 2$. However, if we instead use this inequality in the denominator of first inequality, we can obtain a better growth condition for $n \geq 3$.
\begin{equation}
    x_n \geq \frac{(x_{n-1})^{N+2}}{x_{n-2}} \geq (x_{n-1})^{N+2}(x_{n-1})^\frac{-1}{N+1} = (x_{n-1})^{N+2-\frac{1}{N+1}}
\end{equation}
which is satisfied for $N \geq 1$. This mean any of our sequences will satisfy the criterion provided $n \geq 3$.

\subsection{Proof of Transcendence}
We now have everything we need to prove the main theorem of this section.
\begin{theorem}
Take $(x_k)_{k\in\mathbb{N}}$ to be a sequence satisfying the conditions found in section \ref{conditions}. The Hone series, $S_\infty$ is transcendental.
\end{theorem}
\begin{proof}
Due to the infinite continued fraction expression of $S_\infty$ given in Corollary \ref{correspondence.corollary} we know that it must be irrational. 

For sufficiently large $n$, $(x_k)_{k\in\mathbb{N}}$ satisfies the growth condition $x_{n+1} > x_n^\kappa$ for some $\kappa > 2$. This means we can use Lemma \ref{trancsendence.criterion} to prove that the irrationality of $S_\infty$ implies that is transcendental.
\end{proof}

\section*{Open Questions}
\addcontentsline{toc}{section}{Open Questions}
In this section, we will explore different ways we could have tackled this problem and ideas that could be further developed.

There are many times in this paper where assumptions are made (much/far) earlier than required. Indeed, in Hone's original paper the conditions collated in section \ref{conditions} were dispersed throughout the paper and introduced only when absolutely necessary. In this paper, these have instead been purposely introduced as early as possible so that the reader can have a more complete picture of the structure of the sequence and the series throughout (consumption of) the entire paper. 

There are other ways our series could have been constructed. In a 2017 paper \cite{varona2017continued}, Juan Luis Varona, instead considered taking $\mathcal{S}$ as an alternating sum. Where Hone had constructed an Engel series, this construction generated various Pierce series, later dubbed Varona series. He went on to define the structure of the continued fraction expression of the series, with the structure repeating less frequently every three terms. He then showed that these series also take transcendental value.

There are further properties of the irrational numbers generated by the Hone series to be explored. In a recent paper
\cite{duverney2020irrationality}, Daniel Duverney endeavoured to compute the irrationality exponents of both Hone series and the aforementioned Varona series. That is, they wanted to find the smallest possible $n$ such that \eqref{trancsendence.liouville.2} has finitely many solutions for $c = 1$. He then further generalised this by considering series made of sums of ratios of two distinct recursive sequences.
Note that in the paper $\theta$ is introduced as a new notation for intersequences. Under this notation given the sequence $(x_k)_{k \in \mathbb{N}}$ we would represent the first intersequence, $y_n$, as $\theta x_n$ and the third, $z_n$, as $\theta^2x_n$. This notation is useful when describing the interaction between sequences as in this paper.

\bibliographystyle{plain}
\bibliography{references}

\end{document}
